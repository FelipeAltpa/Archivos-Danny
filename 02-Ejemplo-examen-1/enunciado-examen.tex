\documentclass[11pt,a4paper]{article}
\usepackage[utf8]{inputenc}
\usepackage[spanish]{babel}
\usepackage{geometry}
\usepackage{enumitem}
\usepackage{fancyhdr}
\usepackage{titlesec}
\usepackage{hyperref}
\usepackage{xcolor}

\geometry{margin=2.5cm}
\setlength{\parindent}{0pt}
\setlength{\parskip}{0.5em}

\titleformat{\section}{\normalfont\large\bfseries}{\thesection.}{0.5em}{}
\titleformat{\subsection}{\normalfont\normalsize\bfseries}{\thesubsection}{0.5em}{}

\pagestyle{fancy}
\fancyhf{}
\fancyhead[L]{\small Programación de Software}
\fancyhead[R]{\small Examen - Entrega 19--26 Feb 2025}
\fancyfoot[C]{\thepage}

\hypersetup{colorlinks=true, linkcolor=blue, urlcolor=blue}

\begin{document}

\begin{center}
  \textbf{\large INSTITUCIÓN}\\[0.3em]
  \textbf{\large Programación de Software}\\[0.8em]
  \textbf{\large Enunciado del examen}\\[0.5em]
  \rule{\textwidth}{0.4pt}\\[0.5em]
  POO con Python, PEP8 y flujo Git (ramas protegidas y \texttt{feat})
\end{center}

\vspace{1em}

\section{Objetivo}

Desarrollar un proyecto en Python aplicando \textbf{Programación Orientada a Objetos} (POO), siguiendo las buenas prácticas \textbf{PEP~8} con formateador \textbf{Black} obligatorio, y utilizando \textbf{Git} con un flujo basado en ramas protegidas \texttt{dev}, \texttt{qa} y \texttt{prod} (la rama \texttt{main} no se usa ni se protege) y ramas de aporte \texttt{feat/...}.

\section{Fechas y modalidad}

\begin{itemize}[leftmargin=*]
  \item \textbf{Inicio de entrega:} miércoles 19 de febrero de 2025.
  \item \textbf{Fecha límite:} jueves 26 de febrero de 2025, \textbf{a las 12:00 del mediodía}. Después de esa hora no se tendrán en cuenta nuevos \texttt{push} ni commits; solo se evaluará el estado del repositorio hasta ese momento.
  \item \textbf{Nota:} El jueves 26 de febrero \textbf{no habrá clase}.
  \item \textbf{Entrega formal:} es \textbf{obligatorio} registrar en el archivo Excel que el docente comparte en el equipo/grupo de Teams, junto al nombre de cada estudiante: la \textbf{URL del repositorio GitHub} y el \textbf{nombre de la cuenta de GitHub}. Sin la URL y el usuario de GitHub en el Excel no se considera entregado.
\end{itemize}

\section{Requisitos técnicos}

\subsection{Programación Orientada a Objetos (Python)}
\begin{itemize}[leftmargin=*]
  \item Uso de clases con encapsulamiento (atributos privados/protegidos y \texttt{@property} donde corresponda).
  \item Herencia y reutilización de código (clase base y subclases).
  \item Métodos con responsabilidad clara y tipado con type hints (\texttt{-> None}, \texttt{-> str}, etc.).
  \item Estructura modular (por ejemplo \texttt{src/entities/}, módulos reutilizables).
\end{itemize}

\subsection{Estilo y buenas prácticas (PEP~8) y Black}
\begin{itemize}[leftmargin=*]
  \item \textbf{Uso obligatorio del formateador \texttt{black}:} el código debe estar formateado con Black para garantizar el cumplimiento de PEP~8. Se recomienda ejecutar \texttt{black .} en la raíz del proyecto antes de cada commit.
  \item Nombres en \texttt{snake\_case} para funciones y variables; \texttt{PascalCase} para clases.
  \item Líneas según convención Black (máximo 88 caracteres por defecto).
  \item Docstrings en módulos y funciones relevantes.
  \item Sin \texttt{try/except} innecesarios; validación explícita de entradas cuando se indique.
\end{itemize}

\subsection{Git: ramas protegidas, flujo y nombre del repositorio}
\begin{itemize}[leftmargin=*]
  \item \textbf{Nombre del repositorio:} debe comenzar por \texttt{backend} y luego el nombre del proyecto (ej.: \texttt{backend-banco}, \texttt{backend-sistema-inventario}).
  \item \textbf{Rama \texttt{main}:} \textbf{no} va protegida y \textbf{no} se utiliza en este flujo; el trabajo se hace solo sobre \texttt{dev}, \texttt{qa} y \texttt{prod}.
  \item \textbf{Ramas protegidas (las que sí se usan):}
    \begin{itemize}
      \item \texttt{dev}: desarrollo integrado.
      \item \texttt{qa}: ambiente de pruebas.
      \item \texttt{prod}: producción (o reflejo de entrega final).
    \end{itemize}
  \item \textbf{Ramas de aporte:} cada estudiante debe trabajar en \textbf{ramas \texttt{feat/cambio-que-hizo}}, es decir, el nombre de la rama describe el cambio realizado. Por ejemplo:
    \begin{itemize}
      \item \texttt{feat/clase-cuenta-ahorro}
      \item \texttt{feat/menu-depositos}
      \item Una rama \texttt{feat/<descripcion-del-cambio>} por cada aporte o funcionalidad.
    \end{itemize}
  \item \textbf{Flujo de integración:} los Pull Requests (PR) van desde \texttt{feat/...} hacia cada rama protegida, es decir: \texttt{feat} $\rightarrow$ \texttt{dev}, \texttt{feat} $\rightarrow$ \texttt{qa} y \texttt{feat} $\rightarrow$ \texttt{prod}, según corresponda en cada caso.
  \item \textbf{Requisitos de cada PR:} todo Pull Request debe incluir:
    \begin{itemize}
      \item \textbf{Descripción} del cambio o funcionalidad.
      \item \textbf{Asignación} (assignee).
      \item \textbf{Revisores} (reviewers).
      \item \textbf{Labels} (etiquetas).
    \end{itemize}
  \item \textbf{Evaluación de conversaciones:} se calificarán las conversaciones y revisiones que se realicen en la rama \texttt{dev} (comentarios en PR, respuestas, mejoras sugeridas).
  \item Commits con mensajes claros y en español o inglés (ej.: \textit{``Añade clase CuentaAhorro con interés y meta''}).
\end{itemize}

\section{Criterios de evaluación y ponderación}

La calificación se distribuye de la siguiente forma:

\begin{center}
  \begin{tabular}{|l|c|}
    \hline
    \textbf{Criterio} & \textbf{Ponderación} \\
    \hline
    Programación Orientada a Objetos (POO) & 20\,\% \\
    \hline
    Uso de Git (ramas protegidas, \texttt{feat}, commits, flujo) & 40\,\% \\
    \hline
    Estilo PEP~8 y calidad de código & 20\,\% \\
    \hline
    Entrega, documentación y funcionamiento del programa & 20\,\% \\
    \hline
    \textbf{Total} & \textbf{100\,\%} \\
    \hline
  \end{tabular}
\end{center}

\subsection{Detalle por criterio}
\begin{itemize}[leftmargin=*]
  \item \textbf{POO (20\,\%):} Diseño de clases, herencia, encapsulamiento, type hints y organización en módulos.
  \item \textbf{Git (40\,\%):} Repositorio con nombre \texttt{backend-<proyecto>}; ramas \texttt{dev}, \texttt{qa}, \texttt{prod}; flujo \texttt{feat} $\rightarrow$ \texttt{dev}/\texttt{qa}/\texttt{prod}; PR con descripción, asignación, revisores y labels; conversaciones y revisiones en \texttt{dev}; commits coherentes y mensajes descriptivos.
  \item \textbf{PEP~8 y calidad (20\,\%):} Uso obligatorio de Black, cumplimiento de convenciones de estilo, nombres claros, docstrings y código legible.
  \item \textbf{Entrega y funcionamiento (20\,\%):} URL del repositorio y nombre de cuenta GitHub registrados en el Excel del equipo de Teams antes del cierre; repositorio accesible; \texttt{README} o instrucciones básicas; programa que ejecute según lo pedido.
\end{itemize}

\section{Referencia de ejemplo}

Se puede tomar como referencia el proyecto de la carpeta \texttt{02-Ejemplo-examen-1}: clases \texttt{Cuenta}, \texttt{CuentaAhorro} y \texttt{CuentaCorriente}, menú en \texttt{main.py}, validaciones sin \texttt{try/except} y estructura \texttt{src/entities/}. El enunciado concreto del problema a implementar (dominio del negocio) puede ser indicado por el docente de forma adicional.

\vspace{0.5em}
\hrule
\vspace{0.3em}
\small
\textit{Documento generado para el curso de Programación de Software. Fechas y porcentajes aplicables al periodo indicado.}
\end{document}
